\documentclass[main.tex]{subfiles}


\begin{document}

\begin{spacing}{1.0}


\begin{center}

\textbf{Investigating and Reprogramming RNA Folding with Molecular Probes}\\

\vspace{0.25cm}

by

\vspace{0.25cm}

\textbf{Matthew F. Allan}

\vspace{0.25cm}

Bachelor of Science in Biochemistry and Molecular Biology\\
The Pennsylvania State University, 2018 

\vspace{1cm}

Submitted to the Program of Computational and Systems Biology on 15 August, 2023\\
in Partial Fulfillment of the Requirements for the Degree of\\
Doctor of Philosophy in Computational and Systems Biology

\vspace{1cm}

\end{center}


\noindent
ABSTRACT

\vspace{0.5cm}

Ribonucleic acid (RNA) performs versatile, essential functions in all known organisms and viruses. These functions require the RNA to fold into specific secondary and tertiary structures, and in many cases to switch among multiple structural states. Predicting and experimentally determining RNA structures remains challenging, particularly because of this propensity of one RNA sequence to form a heterogeneous ensemble of structures.

This thesis investigates two related problems in RNA folding. First, a method of nucleic acid origami is developed in which four divergent RNA sequences are reprogrammed by short antisense oligonucleotides (ASOs) to fold into six different 3D wireframe polyhedra. How each type of structural feature in the polyhedra affects the stability of local base pairs is revealed using dimethyl sulfate mutational profiling with sequencing (DMS-MaPseq). Second, the structural ensembles formed by the genome of SARS coronavirus 2 (SARS-CoV-2) are investigated using DMS-MaPseq. In particular, a method for detecting long-range RNA:RNA interactions using ASOs is developed and applied to pinpoint an interaction between the frameshifting stimulation element (FSE) and a sequence of RNA over one kilobase downstream, which occurs in nearly half of the RNA molecules. This technique is expanded to reveal long-range interactions in three additional coronaviruses, suggesting that this type of RNA structure is more common than previously thought.

Overall, this thesis uses ASOs and mutational profiling to reprogram and investigate RNA folding. A wide variety of RNA sequences prove amenable to these techniques, which enable the creation of synthetic RNA structures and the characterization of natural ones including long-range RNA:RNA interactions. The results of this study enable future investigations on developing RNA origami for research and therapeutic applications and on the roles of distant RNA elements in regulating ribosomal frameshifting in coronaviruses.

\end{spacing}

\vspace{1cm}


\noindent
Thesis supervisor: Mark Bathe

\noindent
Title: Professor of Biological Engineering, Massachusetts Institute of Technology

\noindent
Thesis supervisor: Silvi Rouskin

\noindent
Title: Assistant Professor of Microbiology, Harvard Medical School


\end{document}
