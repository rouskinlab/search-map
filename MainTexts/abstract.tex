\documentclass[main.tex]{subfiles}


\begin{document}


\section{Abstract}

RNA molecules perform a diversity of essential functions for which their linear sequences must fold into higher-order structures.
Techniques including crystallography and cryogenic electron microscopy have revealed 3D structures of ribosomal, transfer, and other well-structured RNAs; while chemical probing with sequencing facilitates secondary structure modeling of any RNAs of interest, even within cells.
Ongoing efforts continue increasing the accuracy, resolution, and ability to distinguish coexisting alternative structures.
However, no method can discover and quantify alternative structures with base pairs spanning arbitrarily long distances -- an obstacle for studying viral, messenger, and long noncoding RNAs, which may form long-range base pairs.

Here, we introduce the method of Structure Ensemble Ablation by Reverse Complement Hybridization with Mutational Profiling (SEARCH-MaP) and software for Structure Ensemble Inference by Sequencing, Mutation Identification, and Clustering of RNA (SEISMIC-RNA).
We use SEARCH-MaP and SEISMIC-RNA to discover that the frameshift stimulating element of SARS coronavirus 2 base-pairs with another element 1~kilobase downstream in nearly half of RNA molecules, and that this structure competes with a pseudoknot that stimulates ribosomal frameshifting.
Moreover, we identify long-range base pairs involving the frameshift stimulating element in other coronaviruses including SARS coronavirus 1 and transmissible gastroenteritis virus, and model the full genomic secondary structure of the latter.
These findings suggest that long-range base pairs are common in coronaviruses and may regulate ribosomal frameshifting, which is essential for viral RNA synthesis.
We anticipate that SEARCH-MaP will enable solving many RNA structure ensembles that have eluded characterization, thereby enhancing our general understanding of RNA structures and their functions.
SEISMIC-RNA, software for analyzing mutational profiling data at any scale, could power future studies on RNA structure and is available on GitHub and the Python Package Index.


\end{document}
