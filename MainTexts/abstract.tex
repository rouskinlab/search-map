\documentclass[main.tex]{subfiles}


\begin{document}


\section{Abstract}

In every living organism and virus, RNA molecules perform a diversity of essential functions for which their linear sequences must fold into higher-order structures.
Techniques including crystallography and cryogenic electron microscopy have revealed 3D structures of ribosomal, transfer, and other well-structured RNAs; while chemical probing with sequencing facilitates secondary structure modeling of arbitrary RNAs, even within cells.
Ongoing efforts continue increasing the accuracy, resolution, and ability to distinguish coexisting alternative structures.
However, no method can identify and quantify alternative structures with base pairs spanning arbitrarily long distances, which, as mounting evidence indicates, occur abundantly in the genomes of RNA viruses.
Here, we develop the method of Structure Ensemble Ablation by Reverse Complement Hybridization with Mutational Profiling (SEARCH-MaP) and the data analysis software Structure Ensemble Inference by Sequencing, Mutation Identification, and Clustering of RNA (SEISMIC-RNA).
We use SEARCH-MaP and SEISMIC-RNA to discover that the frameshift stimulating element of SARS coronavirus 2 base-pairs with another element 1~kilobase downstream in nearly half of RNA molecules, and that this structure inhibits the folding of a pseudoknot that stimulates ribosomal frameshifting.
Moreover, we identify long-range base pairs involving the frameshift stimulating element in other coronaviruses including SARS coronavirus 1 and transmissible gastroenteritis virus, and model the full genomic secondary structure of the latter.
These findings suggest that stable long-range base pairs are common in coronaviruses and may regulate ribosomal frameshifting, which is essential for viral RNA synthesis.
We anticipate that SEARCH-MaP and SEISMIC-RNA will facilitate studies on viral, messenger, and long noncoding RNAs -- particularly of alternative structures and long-range base pairing.
Ultimately, the results of many such studies could be collected in a vast database of long RNA structures for training and benchmarking RNA folding algorithms, making possible a true ``AlphaFold for RNA".


\end{document}
