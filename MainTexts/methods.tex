\documentclass[main.tex]{subfiles}


\begin{document}

\section{Methods}
\label{methods}


\subsection{Correcting observer bias due to drop-out of reads}

Let $N$ reads from $K$ clusters align to a reference sequence of length $L$.
Let the proportion of reads whose 5' and 3' ends align, respectively, to coordinates $a$ and $b$ ($1 \le a \le b \le L$) be $\eta_{ab}$ (assuming these proportions are equal for all clusters). Let the mutation rate of base $j$ ($1 \le j \le L$) in cluster $k$ ($1 \le k \le K$) be $\mu_{jk}$.
Let the proportion of cluster $k$ in the ensemble be $\pi_k$.
To express these quantities as probabilities, let $C_k$ be the event that a read comes from cluster $k$; let $E_{ab}$ be the event that a read aligns with 5' and 3' coordinates $a$ and $b$, respectively; let $S_j$ be the event that a read contains position $j$ (i.e. its alignment coordinates $a$ and $b$ satisfy $1 \le a \le j \le b \le L$); let $M_j$ be the event that a read has a mutation at position $j$; and let $G_g$ be the event that a read has no two mutations separated by fewer than $g$ non-mutated bases.

\subsubsection{Deriving mutation rates of reads with no two mutations too close}
\label{calc_p_mut_noclose}

In terms of these events, the total mutation rates ($\mu_{jk}$) are $P(M_j | S_j C_k)$, i.e. the probability that a read would have a mutation at position $j$ given that it contained position $j$ and came from cluster $k$; and the observable mutation rates ($m_{jk}$) are $P(M_j | S_j C_k G_g)$, i.e. the probability that a read would have a mutation at position $j$ given that it contained position $j$, came from cluster $k$, and had no two mutations closer than $g$ bases.
Using these definitions and Bayes' theorem yields a probabilistic formula for $m_{jk}$:
$$m_{jk} = P(M_j | S_j C_k G_g) = P(M_j | S_j C_k) \frac{P(G_g | S_j M_j C_k)}{P(G_g | S_j C_k)} = \mu_{jk} \frac{P(G_g | S_j M_j C_k)}{P(G_g | S_j C_k)}$$

The term $P(G_g | S_j C_k)$ is the probability that a read would have no two mutations closer than $g$ bases given that it contained position $j$ and came from cluster $k$.
It can be computed using $P(G_g | E_{ab} C_k)$ (abbreviated $d_{abk}$): the probability that a read would contain no two mutations closer than $g$ bases given that its 5' and 3' coordinates are $a$ and $b$, repectively ($1 \le a \le b \le L$), and that it came from cluster $k$.
If position $b$ were mutated (probability $\mu_{bk}$), then the read would contain no two mutations closer than $g$ bases if and only if none of the $g$ bases preceding $b$ (i.e. positions $b - g$ to $b - 1$, inclusive) were mutated (probability $\prod_{j'=\max(b-g, a)}^{b-1}(1 - \mu_{j'k})$, abbreviated $w_{\max(b-g, a),b-1,k}$) and two no mutations between positions $a$ and $b - (g + 1)$, inclusive, were too close (probability $d_{a,\max(b-(g+1), a),k})$).
If position $b$ were not mutated (probability $1 - \mu_{bk}$), then the read would contain no two mutations closer than $g$ bases if and only if no mutations between positions $a$ and $b - 1$, inclusive, were too close (probability $d_{a,\max(b-1,a),k}$).
These two possibilities generate a recurrence relation:
$$d_{abk} = \mu_{bk} w_{\max(b-g, a),b-1,k} d_{a,\max(b-(g+1), a),k} + (1 - \mu_{bk}) d_{a,\max(b-1,a),k}$$
The base case is $d_{abk} = 1$ when $a = b$ because such a read would contain one position and thus be guaranteed to have no two mutations too close.
Then, $P(G_g | S_j C_k)$ is the average of $d_{abk}$ over every read that contains position $j$, weighted by the proportions $\eta_{ab}$:
$$P(G_g | S_j C_k) = \frac{\sum_{a=1}^{j}\sum_{b=j}^{L}\eta_{ab}d_{abk}}{\sum_{a=1}^{j}\sum_{b=j}^{L}\eta_{ab}}$$

The term $P(G_g | S_j M_j C_k)$ is the probability that a read would have no two mutations too close given that it contained a mutation at position $j$ and came from cluster $k$.
It can be computed using $P(G_g | M_j E_{ab} C_k)$ (abbreviated $f_{abjk}$): the probability that a read would contain no two mutations too close given that position $j$ is mutated ($1 \le a \le j \le b \le L$), that its 5' and 3' coordinates are $a$ and $b$ (respectively), and that it came from cluster $k$.
Because position $j$ is mutated, having no two mutations too close requires that none of the $g$ bases on both sides of position $j$ be mutated.
The probability that none of the preceding $g$ positions ($j - g$ to $j - 1$) is mutated is $w_{\max(j-g,a),j-1,k}$, while that of the following $g$ positions ($j + 1$ to $j + g$) is $w_{j+1,\min(j+g,b),k}$.
Upstream of the $g$ bases flanking position $j$ (i.e. positions $a$ to $j - (g + 1)$), the probability that no two mutations are too close is $d_{a,\max(j-(g+1),a),k}$; downstream (i.e. positions $j + (g + 1)$ to $b$), the probability is $d_{\min(j+(g+1),b),b,k}$.
Since mutations in these four sections are independent, the probability that the read contains no two mutations too close is the product:
$$f_{abjk} = d_{a,\max(j-(g+1),a),k} w_{\max(j-g,a),j-1,k} w_{j+1,\min(j+g,b),k} d_{\min(j+(g+1),b),b,k}$$
Then, $P(G_g | S_j M_j C_k)$ is the average of $f_{abjk}$ over every read that contains position $j$, weighted by the proportions $\eta_{ab}$.
$$P(G_g | S_j M_j C_k) = \frac{\sum_{a=1}^{j}\sum_{b=j}^{L}\eta_{ab} f_{abjk}}{\sum_{a=1}^{j}\sum_{b=j}^{L}\eta_{ab}}$$

Combining the above results yields an explicit formula for $m_{jk}$:
$$m_{jk} = \mu_{jk} \frac{\sum_{a=1}^{j}\sum_{b=j}^{L}\eta_{ab} f_{abjk}}{\sum_{a=1}^{j}\sum_{b=j}^{L}\eta_{ab}d_{abk}}$$

\subsubsection{Deriving end coordinate proportions of reads with no two mutations too close}
\label{calc_p_ends_noclose}

The total proportions ($\eta_{ab}$) of reads aligned to 5' and 3' coordinates $a$ and $b$, respectively, are $P(E_{ab})$; and the proportions of reads with no two mutations too close that align with coordinates $a$ and $b$ ($e_{abk}$) are $P(E_{ab} | G_g C_k)$.
Note that, while reads are assumed to come from the same distribution of coordinates ($\eta_{ab}$) regardless of their cluster $k$, the observable distribution of coordinates ($e_{abk}$) varies by cluster because P($G_g C_k$) depends on $k$.
Using these definitions and Bayes' theorem yields a probabilistic formula for $e_{abk}$:
$$e_{abk} = P(E_{ab} | G_g C_k) = P(G_g | E_{ab} C_k) \frac{P(E_{ab} | C_k)}{P(G_g | C_k)} = d_{abk} \frac{\eta_{ab}}{P(G_g | C_k)}$$

The term $P(G_g | C_k)$ is the probability that a read would have no two mutations too close given that it came from cluster $k$.
It can be computed as an average of $P(G_g | E_{ab} C_k)$ (i.e. $d_{abk}$) over all coordinates $a$ and $b$ (such that $1 \le a \le b \le L$), weighted by the proportion of each coordinate, $P(E_{ab})$ (i.e. $\eta_{ab}$):
$$P(G_g | C_k) = \frac{\sum_{a=1}^{L} \sum_{b=a}^{L} \eta_{ab} d_{abk}}{\sum_{a=1}^{L} \sum_{b=a}^{L} \eta_{ab}} = \sum_{a=1}^{L} \sum_{b=a}^{L} \eta_{ab} d_{abk}$$
This expression is already normalized because $\sum_{a=1}^{L} \sum_{b=a}^{L} \eta_{ab} = 1$, by definition.

Combining the above results yields an explicit formula for $e_{abk}$:
$$e_{abk} = \frac{\eta_{ab} d_{abk}}{\sum_{a'=1}^{L} \sum_{b'=a'}^{L} \eta_{a'b'} d_{a'b'k}}$$

\subsubsection{Deriving cluster proportions of reads with no two mutations too close}
\label{calc_p_clust_noclose}

The proportion of total reads in cluster $k$ is $\pi_k = P(C_k)$.
The proportion among only reads with no two mutations closer than $g$ bases is
$$p_k = P(C_k | G_g) = P(G_g | C_k) \frac{P(C_k)}{P(G_g)} = \pi_k \frac{\sum_{a=1}^{L} \sum_{b=a}^{L} \eta_{ab} d_{abk}}{P(G_g)}$$
The term $P(G_g)$ is the probability that a read from any cluster would have no two mutations closer than $g$ bases and can be solved for by leveraging that the cluster proportions ($p_k$) must sum to 1:
$$1 = \sum_{k=1}^{K} p_k = \sum_{k=1}^{K} \pi_k \frac{\sum_{a=1}^{L} \sum_{b=a}^{L} \eta_{ab} d_{abk}}{P(G_g)} = \frac{1}{{P(G_g)}} \sum_{k=1}^{K} \pi_k \sum_{a=1}^{L} \sum_{b=a}^{L} \eta_{ab} d_{abk}$$
$$P(G_g) = \sum_{k=1}^{K} \pi_k \sum_{a=1}^{L} \sum_{b=a}^{L} \eta_{ab} d_{abk}$$
The result is an explicit formula for $p_k$:
$$p_k = \frac{\pi_k \sum_{a=1}^{L} \sum_{b=a}^{L} \eta_{ab} d_{abk}}{\sum_{k'=1}^{K} \pi_{k'} \sum_{a=1}^{L} \sum_{b=a}^{L} \eta_{ab} d_{abk'}}$$

\subsubsection{Solving total mutation rates and cluster and coordinate proportions}
\label{calc_params}

The observed mutation rates ($m_{jk}$), end coordinate proportions ($e_{abk}$), and cluster proportions ($p_k$) can be calculated as weighted averages over the $N$ reads with no two mutations too close:
$$m_{jk} = \frac{\sum_{i=1}^{N} z_{ik} x_{ij}}{\sum_{i=1}^{N} z_{ik}}$$
$$e_{abk} = \frac{\sum_{i=1}^{N} z_{ik} y_{abi}}{\sum_{i=1}^{N} z_{ik}}$$
$$p_k = \frac{\sum_{i=1}^{N} z_{ik}}{N}$$
where $x_{ij}$ is $1$ if read $i$ has a mutation at position $j$, otherwise $0$; $y_{abi}$ is $1$ if read $i$ aligns to coordinates $a$ and $b$, otherwise $0$; and $z_{ik}$ is the probability that read $i$ came from cluster $k$.

The original parameters $\mu_{jk}$, $\eta_{abk}$, and $\pi_k$ can be solved by setting the two formula each for $m_{jk}$, $e_{abk}$, and $p_k$ equal to each other, creating a system of equations:
$$\mu_{jk} \frac{\sum_{a=1}^{j}\sum_{b=j}^{L}\eta_{ab} f_{abjk}}{\sum_{a=1}^{j}\sum_{b=j}^{L}\eta_{ab}d_{abk}} = m_{jk} = \frac{\sum_{i=1}^{N} z_{ik} x_{ij}}{\sum_{i=1}^{N} z_{ik}}$$
$$\eta_{ab} \frac{d_{abk}}{\sum_{a'=1}^{L} \sum_{b'=a'}^{L} \eta_{a'b'} d_{a'b'k}} = e_{ab} = \frac{\sum_{i=1}^{N} z_{ik} y_{abi}}{\sum_{i=1}^{N} z_{ik}}$$
$$\pi_k \frac{\sum_{a=1}^{L} \sum_{b=a}^{L} \eta_{ab} d_{abk}}{\sum_{k'=1}^{K} \pi_{k'} \sum_{a=1}^{L} \sum_{b=a}^{L} \eta_{ab} d_{abk'}} = p_k = \frac{\sum_{i=1}^{N} z_{ik}}{N}$$
Solving this entire system at once has proven computationally impractical for all but extremely short sequences.
A more feasible approach is to first solve for $\mu_{jk}$ given an initial guess for $\eta_{ab}$, next solve for $\eta_{ab}$ given the updated $\mu_{jk}$, then solve for $\pi_k$ given the updated $\mu_{jk}$ and $\eta_{ab}$, and iterate until all three sets of parameters converge.

Even assuming every $\eta_{ab}$ is a constant, these equations are still too complex to solve for $\mu_{jk}$ analytically because $d_{abk}$ and $f_{abjk}$ also depend on $\mu_{jk}$ (as well as on other $\mu$ variables).
Thus, every $\mu_{jk}$ is solved for numerically by rearranging each equation to
$$\mu_{jk} \frac{\sum_{a=1}^{j}\sum_{b=j}^{L}\eta_{ab} f_{abjk}}{\sum_{a=1}^{j}\sum_{b=j}^{L}\eta_{ab}d_{abk}} - m_{jk} = 0$$
and applying the Netwon-Krylov method~\cite{Knoll2004} implemented in SciPy~\cite{Virtanen2020}.

Once every $\mu_{jk}$ has been solved for, every $\eta_{ab}$ can be updated.
Because $d_{abk}$ does not depend on $\eta_{ab}$ (except indirectly through the $\mu_{jk}$ parameters, which are now assumed to be constants), each equation can be rearranged to
$$\eta_{ab} = \frac{e_{ab}}{d_{abk}} \sum_{a'=1}^{L} \sum_{b'=a'}^{L} \eta_{a'b'} d_{a'b'k}$$
Leveraging that $\sum_{a=1}^{L} \sum_{b=a}^{L} \eta_{ab} = 1$, by definition, leads to
$$\sum_{a=1}^{L} \sum_{b=a}^{L} \frac{e_{ab}}{d_{abk}} \sum_{a'=1}^{L} \sum_{b'=a'}^{L} \eta_{a'b'} d_{a'b'k} = 1$$
$$\sum_{a'=1}^{L} \sum_{b'=a'}^{L} \eta_{a'b'} d_{a'b'k} = \frac{1}{\sum_{a=1}^{L} \sum_{b=a}^{L} \frac{e_{ab}}{d_{abk}}}$$
and finally a closed-form expression for each $\eta_{ab}$ given $\mu_{jk}$ (and hence $d_{abk}$) and $e_{abk}$:
$$\eta_{ab} = \frac{\frac{e_{ab}}{d_{abk}}}{\sum_{a'=1}^{L} \sum_{b'=a'}^{L} \frac{e_{a'b'}}{d_{a'b'k}}}$$
This equation should theoretically yield the same value of $\eta_{ab}$ for every $k$.
In practice, the values will differ due to inexactness in floating-point arithmetic.
Thus, the consensus value of $\eta_{ab}$ is taken to be the average $\eta_{ab}$ over every $k$, weighted by $\pi_k$:
$$\eta_{ab} = \sum_{k=1}^{K} \pi_k \frac{\frac{e_{ab}}{d_{abk}}}{\sum_{a'=1}^{L} \sum_{b'=a'}^{L} \frac{e_{a'b'}}{d_{a'b'k}}}$$

With updated values of $\mu_{jk}$ and $\eta_{ab}$, $\pi_k$ can also be solved.
The above equations can be rearranged to
$$\pi_k = p_k \frac{\sum_{k'=1}^{K} \pi_{k'} \sum_{a=1}^{L} \sum_{b=a}^{L} \eta_{ab} d_{abk'}}{\sum_{a=1}^{L} \sum_{b=a}^{L} \eta_{ab} d_{abk}}$$
Given that $\sum_{k=1}^{K} \pi_k = 1$, by definition:
$$\sum_{k=1}^{K} p_k \frac{\sum_{k'=1}^{K} \pi_{k'} \sum_{a=1}^{L} \sum_{b=a}^{L} \eta_{ab} d_{abk'}}{\sum_{a=1}^{L} \sum_{b=a}^{L} \eta_{ab} d_{abk}} = 1$$
$$\sum_{k'=1}^{K} \pi_{k'} \sum_{a=1}^{L} \sum_{b=a}^{L} \eta_{ab} d_{abk'} = \frac{1}{\sum_{k=1}^{K} \frac{p_k}{\sum_{a=1}^{L} \sum_{b=a}^{L} \eta_{ab} d_{abk}}}$$
which leads to a closed-form expression for each $\pi_k$ given $\mu_{jk}$ (and hence $d_{abk}$), $\eta_{ab}$, and $p_k$:
$$\pi_k = \frac{\frac{p_k}{ \sum_{a=1}^{L} \sum_{b=a}^{L} \eta_{ab} d_{abk}}}{\sum_{k'=1}^{K} \frac{p_{k'}}{\sum_{a=1}^{L} \sum_{b=a}^{L} \eta_{ab} d_{abk'}}}$$


\subsection{Clustering reads with the expectation-maximization algorithm}

Let $N$ reads from $K$ clusters align to a reference sequence of length $L$.
Let the proportion of reads whose 5' and 3' ends align, respectively, to coordinates $a$ and $b$ ($1 \le a \le b \le L$) be $\eta_{ab}$ (assuming these proportions are equal for all clusters).
Let the mutation rate of base $j$ ($1 \le j \le L$) in cluster $k$ ($1 \le k \le K$) be $\mu_{jk}$.
Let the proportion of cluster $k$ in the ensemble be $\pi_k$.

\subsubsection{Maximization step}

The maximization step updates the parameters ($\mu_{jk}$, $\eta_{ab}$, and $\pi_k$) using the current cluster memberships ($z_{ik}$).
The observed estimates of the parameters $m_{jk}$, $e_{ab}$, and $p_k$ are first computed; then, the underlying parameters $\mu_{jk}$, $\eta_{ab}$, and $\pi_k$ are solved for as described in \ref{calc_params}.

\subsubsection{Expectation step}

The expectation step updates the cluster memberships ($z_{ik}$) and the likelihood function ($L$) using the current parameters ($\mu_{jk}$, $\eta_{ab}$, and $\pi_k$).
Each cluster membership is defined as the probability that read $i$ came from cluster $k$ given its 5'/3' end coordinates ($E_{ab}$) and mutations ($M$) and given that no two mutations are too close ($G_g$): $z_{ik} = P(C_k | E_{ab} M G_g)$.
The likelihood of the model ($L$) is the product of the marginal probability ($L_i$) of observing each read $i$ from any cluster: $L_i = P(E_{ab} M | G_g)$.
Both $L_i$ and $z_{ik}$ can be expressed in terms of the joint probability ($L_{ik} = P(E_{ab} M C_k | G_g)$) of observing each read $i$ from each cluster $k$:
$$L_i = P(E_{ab} M | G_g) = \sum_{k=1}^K P(E_{ab} M C_k | G_g) = \sum_{k=1}^K L_{ik}$$
$$z_{ik} = P(C_k | E_{ab} M G_g) = \frac{P(E_{ab} M C_k G_g)}{P(E_{ab} M G_g)} = \frac{P(E_{ab} M C_k | G_g)}{P(E_{ab} M | G_g)} = \frac{L_{ik}}{L_i}$$

To derive a formula for $L_{ik}$, it can be factored into three parts using the chain rule for probability:
$$L_{ik} = P(E_{ab} M C_k | G_g) = \frac{P(E_{ab} M C_k G_g)}{P(G_g)} = P(M | E_{ab} C_k G_g) P(E_{ab} | C_k G_g) P(C_k | G_g)$$
The first part -- the probability that a read would have the specific mutations $x_{ij}$ given that its 5'/3' end coordinates are $a$ and $b$ (respectively), it comes from cluster $k$, and no two mutations are too close -- is the product over every position $j$ from $a$ to $b$ of the probability of a mutation ($\mu_{jk}$) if read $i$ is mutated at position $j$ ($x_{ij} = 1$), otherwise ($x_{ij} = 0$) the probability of no mutation ($1 - \mu_{jk}$), normalized by the probability that no two mutations would be too close ($d_{abk}$):
$$P(M | E_{ab} C_k G_g) = \frac{1}{d_{abk}} \prod_{j=a}^{b} \mu_{jk}^{x_{ij}} (1 - \mu_{jk})^{(1 - x_{ij})}$$
The second part, $P(E_{ab} | C_k G_g) = e_{abk}$, can be calculated from the parameters $\mu_{jk}$, $\eta_{ab}$, and $\pi_k$, as explained in \ref{calc_p_ends_noclose}.
Likewise, the third part, $P(C_k | G_g) = p_k$, can also be calculated from the parameters, as explained in \ref{calc_p_clust_noclose}.
Combining all parts yields a formula for $L_{ik}$ in terms of the parameters $\mu_{jk}$, $\eta_{ab}$, and $\pi_k$ and of their derived values $d_{abk}$, $e_{abk}$, and $p_k$:
$$L_{ik} = p_k \frac{e_{abk}}{d_{abk}} \prod_{j=a}^{b} \mu_{jk}^{x_{ij}} (1 - \mu_{jk})^{(1 - x_{ij})}$$

The formula for the total likelihood of the model and its parameters follows:
$$L(\mu, \eta, \pi) = \prod_{i=1}^{N} L_i = \prod_{i=1}^{N} \sum_{k=1}^{K} p_k \frac{e_{abk}}{d_{abk}} \prod_{j=a}^{b} \mu_{jk}^{x_{ij}} (1 - \mu_{jk})^{(1 - x_{ij})}$$

\subsection{Screening coronavirus long-range interactions computationally}
\label{screen_lri_comp}

All coronaviruses with reference genomes in the NCBI Reference Sequence Database~\cite{OLeary2016} were searched for using the following query:
\begin{verbatim}
refseq[filter] AND ("Alphacoronavirus"[Organism] OR
                    "Betacoronavirus"[Organism] OR
                    "Gammacoronavirus"[Organism] OR
                    "Deltacoronavirus"[Organism])
\end{verbatim}
The complete record of every reference genome was downloaded both in FASTA format (for the reference sequence) and in Feature Table format (for feature locations).
The location of the frameshift stimulating element (FSE) in each genome was estimated from the feature table, and the nearest instance of \verb|TTTAAAC| was used as the slippery site, using a custom Python script.
The 2,000 nt segment beginning 100 nt upstream of and ending 1,893 nt downstream of the slippery site was used for predicting long-range interactions involving the FSE.
Genomes with ambiguous nucleotides (e.g. \verb|N|) in this segment were discarded.
For each coronavirus genome, up to 100 secondary structure models of the 2,000 nt segment were generated using Fold version 6.3 from RNAstructure~\cite{Mathews2004a} with \verb|-M 100| and otherwise default parameters.
Then, for each position, the fraction of models for the coronavirus in which the base at the position paired with any other base between positions 101 (the first base of the slippery sequence) and 250 was calculated using a custom Python script.
The coronaviruses were clustered by their fraction vectors using the unweighted pair group method with arithmetic mean (UPGMA) and a euclidean distance metric, implemented in Seaborn version 0.11~\cite{Waskom2021} and SciPy version 1.7~\cite{Virtanen2020}.
The resulting hierarchically-clustered heatmap was examined manually to select coronaviruses based on the prominence of potential long-range interactions with the FSE (relatively large fractions far from positions 101-250).

\end{document}
