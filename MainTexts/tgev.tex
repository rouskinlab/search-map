\documentclass[main.tex]{subfiles}


\begin{document}


\subsection{}

To verify that the long-range interaction also forms in live transmissible gastroenteritis virus (TGEV), we infected ST cells with TGEV (two biological replicates) and performed DMS-MaPseq (two technical replicates per biological replicate).
The DMS reactivities were highly reproducible over the whole TGEV genome (\textit{r} = 0.96-0.97, SFIG).
As expected, they differed from those of the 1.8 kb segment \textit{in vitro} (\textit{r} = 0.82, SFIG), showing why it is necessary to verify the long-range interaction in TGEV-infected cells.

First, to determine whether the FSE and the region with which it may interact form alternative structures, we amplified and deeply sequenced these two regions from each sample.
Clustering the reads using SEISMIC-RNA revealed that both regions adopt at least two alternative structures.
The two clusters of the downstream region differed most around positions 1,120-1,140 -- the site of the 3' end of the predicted long-range interaction.
In cluster 1 (63\% of the ensemble), bases 1,129-1,136 (all part of the predicted interaction) had DMS reactivities less than 0.01; while in cluster 2, the DMS reactivities were all greater than 0.01.
This result suggests that cluster 1 corresponds to the state in which the long-range interaction forms.


\end{document}
