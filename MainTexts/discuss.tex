\documentclass[main.tex]{subfiles}


\begin{document}

\section{Discussion}
\label{discussion}

In this work, we developed SEARCH-MaP and SEISMIC-RNA and applied them jointly to detect structural ensembles involving long-range RNA:RNA interactions in SARS-CoV-2 and other coronaviruses. This study is certainly not the first to perturb RNA structure with ASOs, nor even the first to use DMS-MaPseq to quantify the structural changes upon binding ASOs to SARS-CoV-2 RNA~\cite{Zhu2022}. But while this previous study examined local structural perturbations caused by binding an ASO, we show that we can detect changes in the structure at more distant locations in an RNA molecule that interact with the nucleotides bound by an ASO.

A previous study detected two long-range RNA--RNA interactions in the genome of satellite tobacco mosaic virus by binding an ASO (in this case, an LNA 9-mer) to each site, followed by chemical probing~\cite{Archer2013}.
However, SEARCH-MaP and SEISMIC-RNA go further by also determining the mutational profile and proportion of the interaction-formed and -unformed states (Figure \ref{tiles}c, d).
With a collection of candidate structure models, these methods even reveal how adding an ASO ablates specific structures, collapsing the ensemble into one predominant structure (Figure \ref{lnas}d).

Many methods have been developed to find long-range (and intermolecular) RNA--RNA base pairing using crosslinking (with psoralen or a derivative), proximity ligation, and deep sequencing~\cite{Aw2016,Lu2016,Sharma2016,Ziv2018}. These methods require no prior knowledge of RNA--RNA interactions and have no limit to the length of the interactions they can detect. They do, however, suffer from several limitations including inefficient ligation [QUANTIFY, I think I read that less than 5\% of molecules actually ligate] necessitating either enrichment or very deep sequencing, as well as bias towards U-rich sequences. They are not single-molecule techniques, either, meaning that although they can detect mutually exclusive base pairs, they cannot determine which specific alternative structures exist or quantify their proportions, as SEARCH-MaP/SEISMIC-RNA can. There is also no straightforward way to focus on one specific RNA--RNA interaction.

Another method based on applying many ASO "patches" in parallel and reading out the signal with microarray probes has also recently been developed~\cite{Chiang2023}. Like proximity ligation, this method has no limitation to the length of the interactions it could find, yet it is also not a single-molecule technique, meaning that it cannot resolve individual structures in an ensemble.

Could be possible to use SEARCH-MaP to evaluate the accuracy of computational predictions of RNA structures, especially for long-range base pairing.
Also to generate a set of well-characterized secondary structures for developing new tools to predict RNA structures.

SEARCH-MaP bears conceptual similarity to another method, mutate-and-map read out through next-generation sequencing (M2-seq)~\cite{Cheng2017}. Both involve perturbing one region of an RNA molecule (in the case of M2-seq, by pre-installing mutations through error-prone PCR) and measuring the effects on other bases in the RNA using chemical probing. The major differences are the precision and scale of the interactions identified, as well as the throughput. M2-seq can pinpoint interactions down to the resolution of a single base pair, and is thus more precise than SEARCH-MaP. However, DMS-guided RNA structure prediction can propose structure models at single-base-pair resolution, which SEARCH-MaP can validate, and in this way achieve single-base-pair resolution. SEARCH-MaP is also capable of finding interactions over a much longer range because M2-seq requires the interacting bases to be in the same Illumina sequencing read. Within this length limit, one M2-seq experiment can theoretically find all pairwise interactions between bases, while one SEARCH-MaP experiment can find only interactions that involve the region to which the ASOs were hybridized. M2-seq is also limited by the formation of alternative structures. Some methods, such as [CITE something by Rhiju, maybe REEFIT] and DANCE-MaP~\cite{Olson2022}, have been designed to work around this limitation SEARCH-MaP; however, [something by Rhiju] has [this problem], and DANCE-MaP requires extremely high sequencing depth of several million reads [MORE PRECISE]. SEARCH-MaP, by contrast, assumes from the start that the RNA may form alternative structures; for simply detecting long-range interactions, even a 5,000 read depth is sufficient coverage; and for clustering, we have found [SOME LIMIT].

Another limitation of SEARCH-MaP as presented here is that it cannot distinguish between direct and indirect interactions. If RNA segment A interacts with segment B, while B interacts with both segment A and C, then hybridizing an ASO to segment A would perturb the structure of B, which could consequentially perturb the structure of C. Hence, C would appear to interact with A, even though this interaction is indirect, through B. One possible workaround (not shown in this study) would be to mutate or hybridize an ASO to segment B, and then repeat the experiment with hybridizing an ASO to segment A. If the interaction between A and C is direct, then C should still be perturbed even when segment B is incapable of interacting with A or C. But if B mediates an indirect interaction between A and C, then disrupting B should eliminate the apparent interaction between A and C.

Functional long-range interactions up to four kilobases involving an FSE have been been found previously in two plant viruses~\cite{Barry2002,Tajima2011}. In both cases, frameshifting required the long-range interaction, suggesting that this interaction enables negative feedback on synthesis of viral RNA polymerase~\cite{Barry2002}. When polymerase levels are low, the interaction would form and stimulate frameshifting, which is needed to synthesize RNA polymerase. Once the polymerase had accumulated, it would begin to replicate the genomic RNA; in its passage from the genomic 3' end to the 5' end, it would disrupt the 3' side of the long-range interaction, attenuating frameshifting and reducing synthesis of more polymerase.

However, this strategy cannot be the role, if any, of the long-range interactions in coronaviruses. Unlike in the two plant viruses, a long-range interaction is not required to stimulate frameshifting in coronaviruses: numerous studies have shown that even the isolated FSE can cause 15 - 40\% of ribosomes to frameshift~\cite{Bhatt2021,Haniff2020,Kelly2020,Lan2022,Plant2010,YSun2021,KZhang2021}. In coronaviruses, the long-range interaction is not only unnecessary for frameshifting but also may even attenuate it, given that in SARS-CoV-2, the FSE-arch and the frameshift-stimulating pseudoknot seem to be mutually exclusive. Moreover, coronaviruses partition translation and RNA synthesis into two different cellular compartments (the cytosol and the double-membrane vesicles, respectively)~\cite{Wolff2020}, so structural changes induced by RNA polymerases would not be seen by ribosomes.
If any of the long stems existed, they would block the pseudoknot from forming, which suggests a mechanism by which the long-range interaction could regulate the structure -- and possibly frameshifting activity -- of the FSE.
Although we had previously shown that AS1 overlaps and outcompetes PS1~\cite{Lan2022}, AS1 lies upstream of the slippery site and would be unwound by approaching ribosomes, while the long stems lie downstream and would not.

The functions of these long-range interactions involving the FSE in coronaviruses remain mysterious. However, given that they occur in multiple coronaviruses across at least two genera, it seems reasonable that they could play a role in the viral life cycle, possibly by affecting the rate of frameshifting. Further research may reveal new mechanisms of translational regulation in coronaviruses via long-range RNA:RNA interactions.

The emergence of coronavirus disease 2019 (COVID-19) as a pandemic in 2020 spurred many investigations on functional RNA structures in coronaviruses, particularly SARS coronavirus 2 (SARS-CoV-2)~\cite{Rangan2020,Manfredonia2020,Ziv2020,LSun2021,YanZhang2021,Huston2021,Rangan2021,Morandi2021,Yang2021,Lan2022}. Among the more unexpected findings was an RNA:RNA interaction between the frameshifting stimulation element (FSE) and another sequence up to 1,475 nt downstream, which the authors named the FSE-arch~\cite{Ziv2020}. The FSE-arch was detected in infected cells using COMRADES~\cite{Ziv2018} and proposed to comprise three nested long-range RNA:RNA interactions (Figure \ref{search_fig2}a): an outer 38 bp bulged stem spanning coordinates 13,370-14,842 (which encompasses the FSE); a middle 18 bp bulged stem spanning coordinates 13,533-14,673; and an inner 14 bp bulged stem spanning coordinates 13,580-14,552~\cite{Ziv2020}. We had discovered that the FSE folds into at least two alternative structures in infected cells, in roughly equal proportions, and that the predicted structure for one of them resembles the FSE-arch~\cite{Lan2022}. Because computational RNA structure prediction -- even guided by chemical probing data -- is unreliable for long RNA sequences especially~\cite{Aviran2022}, we sought stronger, hypothesis-driven evidence for the existence of the FSE-arch.

\end{document}
