\documentclass[main.tex]{subfiles}


\begin{document}

\section{Introduction}
\label{intro}

The emergence of coronavirus disease 2019 (COVID-19) as a pandemic in 2020 spurred many investigations on functional RNA structures in coronaviruses, particularly SARS coronavirus 2 (SARS-CoV-2)~\cite{Rangan2020,Manfredonia2020,Ziv2020,LSun2021,YanZhang2021,Huston2021,Rangan2021,Morandi2021,Yang2021,Lan2022}. Among the more unexpected findings was an RNA:RNA interaction between the frameshifting stimulation element (FSE) and another sequence up to 1,475 nt downstream, which the authors named the FSE-arch~\cite{Ziv2020}. The FSE-arch was detected in infected cells using COMRADES~\cite{Ziv2018} and proposed to comprise three nested long-range RNA:RNA interactions (Figure \ref{search_fig2}a): an outer 38 bp bulged stem spanning coordinates 13,370-14,842 (which encompasses the FSE); a middle 18 bp bulged stem spanning coordinates 13,533-14,673; and an inner 14 bp bulged stem spanning coordinates 13,580-14,552~\cite{Ziv2020}. We had discovered that the FSE folds into at least two alternative structures in infected cells, in roughly equal proportions, and that the predicted structure for one of them resembles the FSE-arch~\cite{Lan2022}. Because computational RNA structure prediction -- even guided by chemical probing data -- is unreliable for long RNA sequences especially~\cite{Aviran2022}, we sought stronger, hypothesis-driven evidence for the existence of the FSE-arch.


Chemical probing followed by mutational profiling is a common strategy for inferring secondary structures of RNA molecules \cite{Zubradt2016, Siegfried2014}.




Here, we present a method to probe RNA--RNA interactions spanning hundreds to thousands of nucleotides, "Structure Ensemble Ablation by Reverse Complement Hybridization with Mutational Profiling" (SEARCH-MaP).
To compute, compare, and deconvolute data from mutational profiling experiments (including SEARCH-MaP, DMS-MaPseq, and SHAPE-MaP), we introduce the software "Structure Ensemble Inference by Sequencing, Mutation Identification, and Clustering of RNA" (SEISMIC-RNA).





\end{document}
